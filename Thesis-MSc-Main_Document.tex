% #############################################################################
% This is the MAIN DOCUMENT of the Thesis MSc TEMPLATE.
% The content for the Thesis MSc is to be written in separate documents
% located in the folder ./Chapters
%         Aknowledgments.tex
%         Abstract.tex
%         KeyWords.tex
%         Resumo.tex
%         PalavrasChave.tex
%         Acronyms.tex
%         Front_Cover.tex
%         Chapter_1.tex ....Chapter_2 .....
%         ApendixA.tex ... ApendixB.tex...
% -----------------------------------------------------------------------------
% The class "istulthesis" is based on the standard LaTeX 'report' class.
% It can be used for Instituto Superior Tecnico thesis, as it follows the 
% regulations published by the Scientific Council of IST.
% The class defines the document style. 
% IST requires the thesis to be written in Arial or similar. 
% Two arguments in '\documentclass' allow you to define the thesis font: 
% 'Helvetica' and 'AvantGarde', which transforms 
% the default LaTeX font into Helvetica or AvantGarde, respectively.
% #############################################################################
% The document is automatically set for english or portuguese by just selecting
% the MAIN LANGUAGE in file 'Thesis-MSc-Preamble_commands.tex' 
% #############################################################################
% Thesis-MSc
% Version 2.0, August 2018
% BY: Rui Santos Cruz, rui.s.cruz@tecnico.ulisboa.pt
% #############################################################################
% !TEX root = ./main.tex
% -----------------------------------------------------------------------------
%
\documentclass[defaultstyle,10pt,Helvetica]{istulthesis}
%
% -----------------------------------------------------------------------------
% The Preamble document contains all the necessary Packages for typesetting
% Modify it to suit your needs
% -----------------------------------------------------------------------------
% #############################################################################
% Preamble for Thesis-MSc in English or Portuguese
% Required Packages and commands
% --> Please Choose the MAIN LANGUAGE for the Thesis in package BABEL (below)
% !TEX root = ./main.tex
% #############################################################################
% Thesis-MSc
% Version 2.0, August 2018
% BY: Rui Santos Cruz, rui.s.cruz@tecnico.ulisboa.pt
% #############################################################################
%
% -----------------------------------------------------------------------------
% PACKAGES ucs, utf8x, babel, iflang:
% -----------------------------------------------------------------------------
% The 'ucs' package provides support for using UTF-8 in LaTeX documents. 
% However in most situations it is not required.
\usepackage{ucs}
% The 'utf8x' package contains support for using UTF-8 as input encoding. 
\usepackage[utf8x]{inputenc}
% The 'babel' package may correct some hyphenation issues of LaTeX. 
% Select your MAIN LANGUAGE for the Thesis with the 'main=' option.
\usepackage[main=english,portuguese]{babel}
% The 'iflang' package is used to help determine the language being used. 
\usepackage{iflang}

% -----------------------------------------------------------------------------
% PACKAGE scrbase:
% -----------------------------------------------------------------------------
% The 'scrbase' package is used to help redefining document structure.
\usepackage{scrbase}
% -----------------------------------------------------------------------------
% PACKAGE mathtools, amsmath, amsthm, amssymb, amsfonts, nicefrac:
% -----------------------------------------------------------------------------
% These packages are typically required. 
% Among many other things they add the possibility to put symbols in bold
% by using \boldsymbol (not \mathbf); defines additional fonts and symbols;
% adds the \eqref command for citing equations.
\usepackage{mathtools, amsmath, amsthm, amssymb, amsfonts}
\usepackage{nicefrac}
%
% -----------------------------------------------------------------------------
% PACKAGE tikz:
% -----------------------------------------------------------------------------
% Tikz  for creating graphics programmatically.
\usepackage{tikz}
\usetikzlibrary{shapes.geometric, arrows, positioning}
% -----------------------------------------------------------------------------
% PACKAGES array, booktabs, multirow, colortbl, ctable, spreadtab:
% -----------------------------------------------------------------------------
% These packages are most usefull for advanced tables. 
% 'multirow' allows to join rows throuhg the command \multirow which works
% similarly with the command \multicolumn.
% The 'colortbl' package allows to color the table (foreground and background)
% The 'ctable' package provides commands to easily typeset centered or left or
% right aligned tables.
% The package 'booktabs' provide some additional commands to enhance
% the quality of tables
% The 'longtable' package is only required when tables extend beyond the length
% of one page, which typically does not happen and should be avoided
\usepackage{array}
\usepackage{booktabs}
\usepackage{multirow}
\usepackage{colortbl}
\usepackage{ctable}
\usepackage{spreadtab}
\usepackage{longtable}
%
% -----------------------------------------------------------------------------
% PACKAGES graphicx, subfigure:
% -----------------------------------------------------------------------------
% The package 'graphicx' supports formats PNG and JPG.
% Package 'subfigure' allows to place figures within figures with own caption. 
% For each of the subfigures use the command \subfigure.
\usepackage{graphicx}
\usepackage[hang,small,bf,tight]{subfigure}
%
% -----------------------------------------------------------------------------
% PACKAGE caption:
% -----------------------------------------------------------------------------
% The 'caption' package offers customization of captions in floating 
% environments such figure and table
% \usepackage[hang,small,bf]{caption}
\usepackage[format=hang,labelfont=bf,font=small]{caption} 
% the following customization adds vertical space between caption and the table
\captionsetup[table]{skip=10pt}
%
% -----------------------------------------------------------------------------
% PACKAGE algorithmic, algorithm, algorithm2e:
% -----------------------------------------------------------------------------
% These packages are required if you need to describe an algorithm.
% The preference is for using 'algorithm2e'
%\usepackage{algorithmic}
%\usepackage[chapter]{algorithm}
\usepackage[ruled,vlined,algochapter,norelsize,\languagename]{algorithm2e}
%
% -----------------------------------------------------------------------------
% PACKAGE listings
% -----------------------------------------------------------------------------
% These packages are required if you need to list code snippets.
\usepackage{listings}
% Nicely syntax highlighted m-code in LaTeX documents with stylefile mcode.sty
% http://www.mathworks.com/matlabcentral/fileexchange/8015-m-code-latex-package
\usepackage[numbered]{./tables_and_code/mcode}
%
% -----------------------------------------------------------------------------
% Re-define listings captions and titles based on language.
\newcaptionname{portuguese}{\lstlistingname}{Listagem} % Listings CAPTIONS
\newcaptionname{portuguese}{\lstlistlistingname}{Listagens} % LIST of LISTINGS
%
% -----------------------------------------------------------------------------
% PACKAGE csquotes
% -----------------------------------------------------------------------------
% Quotation helper package
\usepackage{csquotes}
%
% -----------------------------------------------------------------------------
% PACKAGE todonotes
% -----------------------------------------------------------------------------
% Create TODO Notes in text
% The notes can be made invisible by just using the 'disable' option:
\usepackage[textwidth=2cm, textsize=small]{todonotes}
%\usepackage[textwidth=2cm, textsize=small, disable]{todonotes}
\setlength{\marginparwidth}{2cm}
%
% -----------------------------------------------------------------------------
% PACKAGE changes
% -----------------------------------------------------------------------------
% Track changes in document (changes in pdf preview).
%% Use "final" option to make all tracking markups invisible.
%\usepackage[authormarkup=superscript,authormarkuptext=id,markup=underlined,ulem={ULforem,normalbf},final]{changes}
\usepackage[authormarkup=superscript,authormarkuptext=id,markup=underlined,ulem={ULforem,normalbf}]{changes}
% commands:
% \added[id=xx]{text}
% \deleted[id=xx]{text}
% \replaced[id=xx]{deleted text}{added text}
% -----------------------------------------------------------------------------
% PACKAGES xcolor, color
% -----------------------------------------------------------------------------
% These packages are required for list code snippets.
\usepackage{xcolor}
\usepackage{color}
% The following special color definitions are used in the IST Thesis
\definecolor{forestgreen}{RGB}{34,139,34}
\definecolor{orangered}{RGB}{239,134,64}
\definecolor{lightred}{rgb}{1,0.4,0.5}
\definecolor{orange}{rgb}{1,0.45,0.13}	
\definecolor{darkblue}{rgb}{0.0,0.0,0.6}
\definecolor{lightblue}{rgb}{0.1,0.57,0.7}
\definecolor{gray}{rgb}{0.4,0.4,0.4}
\definecolor{lightgray}{rgb}{0.95, 0.95, 0.95}
\definecolor{darkgray}{rgb}{0.4, 0.4, 0.4}
\definecolor{editorGray}{rgb}{0.95, 0.95, 0.95}
\definecolor{editorOcher}{rgb}{1, 0.5, 0} % #FF7F00 -> rgb(239, 169, 0)
\definecolor{chaptergrey}{rgb}{0.6,0.6,0.6}
\definecolor{editorGreen}{rgb}{0, 0.5, 0} % #007C00 -> rgb(0, 124, 0)
\definecolor{olive}{rgb}{0.17,0.59,0.20}
\definecolor{brown}{rgb}{0.69,0.31,0.31}
\definecolor{purple}{rgb}{0.38,0.18,0.81}
%
% -----------------------------------------------------------------------------
% PACKAGE setspace:
% ----------------------------------------------------------------------------
% Provides support for setting the spacing between lines in a document. 
% Package options include single spacing, one half spacing, and double spacing. 
% Alternatively the spacing can be changed as required with:
% \singlespacing, \onehalfspacing, and \doublespacing commands
\usepackage{setspace}
%
% -----------------------------------------------------------------------------
% PACKAGE paralist
% -----------------------------------------------------------------------------
% This package provides the 'inparaenum' environment for inline lists
\usepackage{paralist}
% usage:
% \begin{inparaenum}[(a)]
% \item bla
% \item bla, bla
% \end{inparaenum}
% -----------------------------------------------------------------------------
% PACKAGE cite:
% -----------------------------------------------------------------------------
% The 'cite' package will result in citation numbers being automatically
% sorted and properly "ranged". i.e.,
% [1], [2], [5]--[7], [9]
\usepackage{cite}
%
% -----------------------------------------------------------------------------
% PACKAGE acronym:
% -----------------------------------------------------------------------------
% The package 'acronym' garantees that all acronyms definitions are 
% given at the first usage. 
% IMPORTANT: do not use acronyms in titles/captions; otherwise the definition 
% will appear on the table of contents.
\usepackage[printonlyused]{acronym}
%
% -----------------------------------------------------------------------------
% PACKAGE hyperref
% -----------------------------------------------------------------------------
% Set links for references and citations in document
\usepackage{hyperref}
% pre-configuration of hyperref
\hypersetup{ colorlinks=true,
             citecolor=cyan,
             linkcolor=darkgray,
             urlcolor=teal,
             breaklinks=true,
             bookmarksnumbered=true,
             bookmarksopen=true,
             pdftitle=\@title, % THESIS TITLE
             pdfauthor=\@author,  % YOUR NAME
             pdfcreator=\@author,   % YOUR NAME
}
%
% -----------------------------------------------------------------------------
% PACKAGE url:
% -----------------------------------------------------------------------------
% Provides better support for handling and breaking URLs.
\usepackage{url} 
%
% -----------------------------------------------------------------------------
% PACKAGE Cleveref:
% -----------------------------------------------------------------------------
% Clever Referencing of document parts
% Note: portuguese is supported through "brazilian" option
\usepackage[\IfLanguageName{english}{english}{brazilian}]{cleveref}
%
% -----------------------------------------------------------------------------
% PACKAGE enumitem:
% -----------------------------------------------------------------------------
%For enhanced enumeration of lists
%\usepackage{enumitem}
\usepackage[shortlabels]{enumitem}
\setlist[description]{leftmargin=\parindent,labelindent=\parindent,itemsep=1pt,parsep=0pt,topsep=0pt}
%
% #############################################################################
% GLOBAL FORMATTING OF THE THESIS DOCUMENT before using FANCY stuff
% Set paragraph counter to alphanumeric mode
\renewcommand{\theparagraph}{\Alph{paragraph}~--}
\hoffset 0in
\voffset 0in
\oddsidemargin 0 cm
\evensidemargin 0 cm
\marginparsep 0in
\topmargin -0.25cm
\textwidth 16 cm
\textheight 22.4 cm
\makeatletter
% package indentfirst says \let\@afterindentfalse\@afterindenttrue
% and we revert this modification, reinstating the original definitio
% of \@afterindentfalse
\def\@afterindentfalse{\let\if@afterindent\iffalse}
\makeatother
% -----------------------------------------------------------------------------
% PACKAGE fancyhdr:
% -----------------------------------------------------------------------------
% The fancyhdr macro package allows to customize page headers and footers.
\usepackage{fancyhdr}
\pagestyle{fancy}
\renewcommand{\chaptermark}[1]{\markboth{\thechapter.\ #1}{}}
\renewcommand{\sectionmark}[1]{\markright{\thesection\ #1}}
\fancyhead{}
\renewcommand{\headrulewidth}{0.0pt}
\renewcommand{\footrulewidth}{0.0pt}
\addtolength{\headheight}{2pt} % make space for the rule
\fancypagestyle{plain}{%
   \fancyhead{} % get rid of headers
   \renewcommand{\headrulewidth}{0pt} % and the line
   \renewcommand{\footrulewidth}{0pt}
}
\fancypagestyle{blank}{%
   \fancyhf{} % get rid of headers and footers
   \renewcommand{\headrulewidth}{0pt} % and the line
   \renewcommand{\footrulewidth}{0pt}
}
\fancypagestyle{abstract}{%
   \fancyhead{}
   \renewcommand{\headrulewidth}{0pt}
   \renewcommand{\footrulewidth}{0.0pt}
}
\fancypagestyle{document}{%
	\fancyhead{}
	\renewcommand{\headrulewidth}{0.5pt}
	\renewcommand{\footrulewidth}{0.5pt}
	\addtolength{\headheight}{2pt} % make space for the rule
}
\setcounter{secnumdepth} {5}
\setcounter{tocdepth} {5}
\renewcommand{\thesubsubsection}{\thesubsection.\Alph{subsubsection}}
\renewcommand{\subfigtopskip}{0.3 cm}
\renewcommand{\subfigbottomskip}{0.2 cm}
\renewcommand{\subfigcapskip}{0.3 cm}
\renewcommand{\subfigcapmargin}{0.2 cm}
%
% -----------------------------------------------------------------------------
% PACKAGE minitoc:
% -----------------------------------------------------------------------------
% Package 'minitoc' creates a mini-table of contents (a “minitoc”) at 
% the beginning of each chapter of a document.
% This packages are required for the \fancychapter configuration
\usepackage{minitoc}
\setcounter{minitocdepth}{1}
\setlength{\mtcindent}{24pt}
\renewcommand{\mtcfont}{\small\rm}
\renewcommand{\mtcSfont}{\small\bf}
\renewcommand*{\kernafterminitoc}{\kern0.\baselineskip\kern0.ex}
\mtcselectlanguage{\languagename} 
% Now prepare the MINITOC
\def\boxedverbatim{%
  \def\verbatim@processline{%
    {\setbox0=\hbox{\the\verbatim@line}%
    \hsize=\wd0 \the\verbatim@line\par}}%
  \@minipagetrue%%%DPC%%%
  \@tempswatrue%%%DPC%%%
  \setbox0=\vbox\bgroup\vspace*{0.2cm}\footnotesize\verbatim
}
\def\endboxedverbatim{%
  \endverbatim
  \unskip\setbox0=\lastbox %%%DPC%%%
  \hspace*{0.2cm}
  \vspace*{-0.2cm}
  \egroup
  \fbox{\box0}% <<<=== change here for centering,...
}
% Now prepare the CHAPTER Number
\newcommand*{\chapnumfont}{%
%   \usefont{T1}{\@defaultcnfont}{b}{n}\fontsize{100}{130}\selectfont%
  \usefont{T1}{pbk}{b}{n}
  \fontsize{150}{130}
  \selectfont
  \color{chaptergrey}
}
\makeatletter
\def\@makechapterhead#1{%
  \vspace*{50\p@}%
  {\parindent \z@ \raggedright \normalfont
    {\chapnumfont\ifnum \c@secnumdepth >\m@ne
%         \huge\bfseries \@chapapp\space \thechapter
        \raggedleft\bfseries \thechapter
        \par\nobreak
        \vskip 20\p@
    \fi}
    \interlinepenalty\@M
    {\raggedleft\Huge \bfseries #1\par\nobreak}
    \vskip 40\p@
  }}
\makeatother
% Now put it all together as a command \fancychapter
\newcommand{\fancychapter}[1]{\chapter{#1}\vfill\minitoc\pagebreak}
%
% #############################################################################
% ADDITIONAL COMMANDS AND CONFIGURATIONS
% #############################################################################
% This commmand allows to place horizontal lines with a custom width... 
% replaces the standard hline command
\newcommand{\hlinew}[1]{%
  \noalign{\ifnum0=`}\fi\hrule \@height #1 \futurelet
   \reserved@a\@xhline}
%   
% -----------------------------------------------------------------------------
% This command defines some marks... USEFUL FOR TABLES.
\def\Mark#1{\raisebox{0pt}[0pt][0pt]{\textsuperscript{\footnotesize\ensuremath{\ifcase#1\or *\or \dagger\or \ddagger\or%
    \mathsection\or \mathparagraph\or \|\or **\or \dagger\dagger%
    \or \ddagger\ddagger \else\textsuperscript{\expandafter\romannumeral#1}\fi}}}}
%
% -----------------------------------------------------------------------------
% The following configurations are used for LISTINGS of certain languages
\lstdefinestyle{XML} {
	language=XML,
	extendedchars=true, 
	breaklines=true,
	breakatwhitespace=true,
	emph={},
	emphstyle=\color{red},
	basicstyle=\small,
	xleftmargin=17pt,
	columns=fullflexible,
	commentstyle=\color{gray}\upshape,
	morestring=[b][\color{brown}]",
	morecomment=[s]{<?}{?>},
	morecomment=[s][\color{forestgreen}]{<!--}{-->},
	keywordstyle=\color{orangered},
	stringstyle=\ttfamily\color{black},
	% stringstyle=\ttfamily\color{black}\normalfont,
	tagstyle=\color{blue},
	% tagstyle=\color{darkblue}\bf,
	morekeywords={asn,action,addrType,abilityNAT,audioSampleRate,audiChannels,,bandwidth,bitmapSize,bitRate,connection,codecs,concurrentLinks,dependency,duration,frameRate,from,height,ip,id,lang,mimeType,onlineTime,peerMode,port,priority,peerProtocol,property,release,to,tier,type,transactionID,url,uploadBWlevel,version,width},
	otherkeywords={attribute,xmlns,schemaLocation,PresentationType,availabilityStartTime,availabilityEndTime,minimumUpdatePeriod,minBufferTime,UpdateTime},
}
% ----------------------------------------------------------------------------
\lstdefinelanguage{Assembler}{
	morecomment=[l];,
	keywords={ADD,ADDC,SUB,SUBB,CMP,MUL,DIV,MOD,NEG,AND,OR,NOT,XOR,TEST,BIT,SET,EI,EI0,EI1,EI2,EI3,SETC,EDMA,CLR,DI,DI0,DI1,DI2,DI3,CLRC,SHR,SHL,SHRA,SHLA,ROR,ROL,RORC,ROLC,MOV,MOVB,MOVBS,MOVP,MOVL,MOVH,SWAP,PUSH,POP,JZ,JNZ,JN,JNN,JP,JNP,JC,JNC,JV,JNV,JEQ,JNE,JLT,JLE,JGT,JGE,JA,JAE,JB,JBE,JMP,CALL,CALLF,RET,RETF,SWE,RFE,NOP},
	morekeywords={EQU,TABLE,WORD,STRING,PLACE},
} 
% ----------------------------------------------------------------------------
\lstdefinestyle{coloredASM}{
	language=Assembler,
	extendedchars=false,
	breaklines=true,
	tabsize=2,
	numberstyle=\tiny,
	numbers=left,
	breakatwhitespace=true,
	emph={},
	emphstyle=\color{red},
	fontadjust=true,
	basicstyle=\small\ttfamily,
	% basicstyle=\footnotesize\ttfamily,
	columns=fixed,
	xleftmargin=17pt,
	framexleftmargin=17pt,
	framexrightmargin=5pt,
	framexbottommargin=4pt,
	commentstyle=\color{forestgreen}\upshape,
	morestring=[b][\color{brown}]",
	keywordstyle=\color{darkblue},
	stringstyle=\ttfamily\color{black},
	literate={á}{{\'a}}1 {ã}{{\~a}}1 {â}{{\^a}}1 {é}{{\'e}}1 {É}{{\'E}}1 {ê}{{\^e}}1 {õ}{{\~o}}1 {ó}{{\'o}}1 {í}{{\'i}}1 {ç}{{\c{c}}}1 {Ç}{{\c{C}}}1,
}    
% ----------------------------------------------------------------------------
\lstdefinelanguage{CSS}{
	sensitive=true,
	morecomment=[l]{//},
	morecomment=[s]{/*}{*/},
	morestring=[b]',
	morestring=[b]",
	alsoletter={:},
	alsodigit={-},
	keywords={color,background-image:,margin,padding,font,weight,display,position,top,left,right,bottom,list,style,border,size,white,space,min,width, transition:, transform:, transition-property, transition-duration, transition-timing-function}
}
% ----------------------------------------------------------------------------
% JavaScript
\lstdefinelanguage{JavaScript}{
	morecomment=[s]{/*}{*/},
	morecomment=[l]//,
	morestring=[b]",
	morestring=[b]',
	morekeywords={typeof, new, true, false, catch, function, return, null, catch, switch, var, if, in, while, do, else, case, break}
}
% ----------------------------------------------------------------------------
\lstdefinelanguage{HTML5}{
	language=html,
	sensitive=true,	
	alsoletter={<>=-},	
	morecomment=[s]{<!-}{-->},
	tag=[s],
	otherkeywords={
	% General
	>,
	% Standard tags
	<!DOCTYPE,
	</html, <html, <head, <title, </title, <style, </style, <link, </head, <meta, />,
	% body
	</body, <body,
	% Divs
	</div, <div, </div>, 
	% Paragraphs
	</p, <p, </p>,
	% scripts
	</script, <script,
	% More tags...
	<canvas, /canvas>, <svg, <rect, <animateTransform, </rect>, </svg>, <video, <source, <iframe, </iframe>, </video>, <image, </image>, <header, </header, <article, </article},
	ndkeywords={
	% General
	=,
	% HTML attributes
	charset=, src=, id=, width=, height=, style=, type=, rel=, href=,
	% SVG attributes
	fill=, attributeName=, begin=, dur=, from=, to=, poster=, controls=, x=, y=, repeatCount=, xlink:href=,
	% properties
	margin:, padding:, background-image:, border:, top:, left:, position:, width:, height:, margin-top:, margin-bottom:, font-size:, line-height:,
	% CSS3 properties
	transform:, -moz-transform:, -webkit-transform:,
	animation:, -webkit-animation:,
	transition:,  transition-duration:, transition-property:, transition-timing-function:,
	}
}
% ----------------------------------------------------------------------------
\lstdefinestyle{htmlcssjs} {%
	% General design
	backgroundcolor=\color{editorGray},
		fontadjust=true,
	basicstyle=\small\ttfamily,   
	frame=b,
	% line-numbers
	xleftmargin={0.75cm},
	numbers=left,
	stepnumber=1,
	firstnumber=1,
	numberfirstline=true,	
	% Code design
	identifierstyle=\color{black},
	keywordstyle=\color{blue}\bfseries,
	ndkeywordstyle=\color{editorGreen}\bfseries,
	stringstyle=\color{editorOcher}\ttfamily,
	commentstyle=\color{brown}\ttfamily,
	% Code
	language=HTML5,
	alsolanguage=JavaScript,
	alsodigit={.:;},	
	tabsize=2,
	showtabs=false,
	showspaces=false,
	showstringspaces=false,
	extendedchars=true,
	breaklines=true,
	% German umlauts
	literate=%
	{Ö}{{\"O}}1
	{Ä}{{\"A}}1
	{Ü}{{\"U}}1
	{ß}{{\ss}}1
	{ü}{{\"u}}1
	{ä}{{\"a}}1
	{ö}{{\"o}}1
}
% ----------------------------------------------------------------------------
\lstdefinestyle{py} {%
	language=python,
	literate=%
	*{0}{{{\color{lightred}0}}}1
	{1}{{{\color{lightred}1}}}1
	{2}{{{\color{lightred}2}}}1
	{3}{{{\color{lightred}3}}}1
	{4}{{{\color{lightred}4}}}1
	{5}{{{\color{lightred}5}}}1
	{6}{{{\color{lightred}6}}}1
	{7}{{{\color{lightred}7}}}1
	{8}{{{\color{lightred}8}}}1
	{9}{{{\color{lightred}9}}}1,
	basicstyle=\small\ttfamily,
	numbers=left,
	% numberstyle=\tiny,
	% stepnumber=2,
	numbersep=5pt,
	tabsize=4,
	extendedchars=true,
	breaklines=true,
	keywordstyle=\color{blue}\bfseries,
	frame=b,
	commentstyle=\color{brown}\itshape,
	stringstyle=\color{editorOcher}\ttfamily,
	showspaces=false,
	showtabs=false,
	xleftmargin=17pt,
	framexleftmargin=17pt,
	framexrightmargin=5pt,
	framexbottommargin=4pt,
	backgroundcolor=\color{lightgray},
	showstringspaces=false,
}
%
% #############################################################################
% #############################################################################
\begin{document}
%
% Add PDF bookmark 
\pdfbookmark[0]{Titlepage}{Title}
% #############################################################################
% DEFINE THE Front Cover Page of Thesis-MSc
% !TEX root = ./main.tex
% #############################################################################
% Thesis-MSc
% Version 2.0, August 2018
% BY: Rui Santos Cruz, rui.s.cruz@tecnico.ulisboa.pt
% #############################################################################
%
% REQUIRED LOGO:
% The university logo image: arguments correspond to {left}{top} position. 
% IST rules determine the position to be be 2cm from top, left page edge
\univlogo{2cm}{2cm}{./Images/IST_A_RGB_POS}
% OPTIONAL IMAGE:
% The thesis image: arguments are the start position in the page.
% You can change the image for your thesis, replacing the image name:
%\thesislogo{2.5cm}{6cm}{./Images/thesis_logo}
\thesislogo{2.5cm}{6cm}{./Images/tecnico-lisboa}
%
% -----------------------------------------------------------------------------
% REQUIRED: Thesis TITLE
\title{Motion Planning for Cooperative Autonomous Robots using Optimisation Tools}
% OPTIONAL: Thesis SUBTITLE
%\subtitle{This is the Thesis Subtitle if Necessary}
%
% -----------------------------------------------------------------------------
% REQUIRED: Author
% Author full Name
\author{Thomas David Pamplona Berry}
%
% -----------------------------------------------------------------------------
% The official name of the course/degree. Please chose portuguese or english
% un-comment the line corresponding to your degree.
% You can add a degree name using this construct
%
\degree{Electrical And Computer Engineering}
%
% -----------------------------------------------------------------------------
% REQUIRED: The SUPERVISOR(s) - maximum of two
\supervisor{Prof. António Manuel dos Santos Pascoal}
%
% -----------------------------------------------------------------------------
% REQUIRED: Date of examination
% Insert the Date of the Thesis discussion (format is MONTH and YEAR)
\date{December 2020}
%
% -----------------------------------------------------------------------------
% The following command define the author colors for Tracking Changes in doc
\definechangesauthor[color=forestgreen]{MN}
\definechangesauthor[color=blue]{JO}
\definechangesauthor[color=red]{PT}

% -----------------------------------------------------------------------------
% Place 'false' when delivering the draft version of the thesis.
% The committee members should not be printed for the draft version. 
% Place 'true' after the Examination Committee has accepted the thesis as final
%\finalthesis{true}
\finalthesis{false}
%
% -----------------------------------------------------------------------------
% The members of the Examination Committee
\chairperson{Prof. Name of the Chairperson}
\vogalone{Prof. Name of First Committee Member}
\vogaltwo{Dr. Name of Second Committee Member}
\vogalthree{Eng. Name of Third Committee Member}
%
% -----------------------------------------------------------------------------
% Please DO NOT MODIFY the following lines.
% print the titlepage
\maketitle
\clearpage
\thispagestyle{empty}

\cleardoublepage
%
% -----------------------------------------------------------------------------
% PAGE NUMBERING FOR INDEXING MATTER in ROMAN
\setcounter{page}{1} \pagenumbering{roman}
\baselineskip 18pt % line spacing: -12pt for single spacing
                   %               -18pt for 1 1/2 spacing
                   %               -24pt for double spacing
% -----------------------------------------------------------------------------
% THE ACKNOWLEGMENTS
\pdfbookmark[0]{Acknowledgments}{acknowledgments}
\begin{acknowledgments}
	% #############################################################################
% Agradecimentos / Acknowledgments
% !TEX root = ../main.tex
% #############################################################################

I would like to thank my parents and sister for their encouragement, caring and support over all these years, for always being there for me through thick and thin and without whom this project would not be possible.

I would like to acknowledge my dissertation supervisor, Prof. Pascoal for his insight, support and sharing of knowledge that has made this thesis possible.

% I would also like to thank Pitchaporn Paitoing, in particualar, for her help and friendship.



\end{acknowledgments}
%
% -----------------------------------------------------------------------------
% THE ABSTRACT
\pdfbookmark[0]{Abstract}{Abstract}
\begin{abstract}
	% #############################################################################
% Abstract Text
% !TEX root = ../main.tex
% #############################################################################
% use \noindent in firts paragraph

\par This thesis presents methods for cooperative motion control for control of fleets of underwater autonomous robots. Good, robust, and fast-to-calculate methods for multiple underwater vehicle motion control is important to take into account inter-vehicle constraints, environmental constraints, energy consumption and mission duration. A suite of programs have been developed using Matlab that generate approximations of optimal trajectories for vehicles in a 2D space. %These programs will be extended for calculating trajectories for multiple cooperative robots. %The programs will eventually be implemented in autonomous robots of the medusa class from ISR-Técnico.


\end{abstract}
\begin{keywords}
	% #############################################################################
% English Keywords
% !TEX root = ../main.tex
% #############################################################################
% use \noindent in firts paragraph
\noindent Motion Planning, Cooperative Autonomous Vehicles, optimization, Bezier curves, Differentially Flat systems.

\end{keywords}
\clearpage
\thispagestyle{empty}
%% If Printing on DOUBLE SIDED pages, the second page should be white.
%% Otherwise, comment the following command:
\cleardoublepage
%
% -----------------------------------------------------------------------------
% O RESUMO
\pdfbookmark[0]{Resumo}{Resumo}
\begin{resumo}
	% #############################################################################
% RESUMO em Português
% !TEX root = ../main.tex
% #############################################################################
% use \noindent in firts paragraph
\noindent 
\end{resumo}
\begin{palavraschave}
	% #############################################################################
% Portuguese Keywords
% !TEX root = ../main.tex
% #############################################################################
% use \noindent in firts paragraph
\noindent Planeamento, Veículos Autonomos Cooperativos, Optimização, Curvas de Bezier, Sistemas Differentially Flat.

\end{palavraschave}
\clearpage
\thispagestyle{empty}
%% If Printing on DOUBLE SIDED pages, the second page should be white.
%% Otherwise, comment the following command:
\cleardoublepage
%
% -----------------------------------------------------------------------------
% This is required for the Fancy Chapters with minitoc
\dominitoc
\dominilof
\dominilot
% -----------------------------------------------------------------------------
% Lists of Contents
\renewcommand{\baselinestretch}{1}
\pdfbookmark[0]{Contents}{toc}
\tableofcontents
%\contentsline{chapter}{References}{\pageref{bib}}
% If Printing on DOUBLE SIDED pages, the second page should be white.
% Otherwise, comment the following command:
\cleardoublepage
% reposition baseline
\renewcommand{\baselinestretch}{1.5}
% -----------------------------------------------------------------------------
% List of Figures
\pdfbookmark[1]{List of Figures}{lof}
\listoffigures
\cleardoublepage
% -----------------------------------------------------------------------------
\begingroup 
    \let\clearpage\relax
    \let\cleardoublepage\relax
    \let\cleardoublepage\relax
% List of Tables
\pdfbookmark[1]{List of Tables}{lot}
\listoftables
% If Printing on DOUBLE SIDED pages, the second page should be white.
% Otherwise, comment the following command:
\let\cleardoublepage\relax
%\cleardoublepage
% -----------------------------------------------------------------------------
% List of Algorithms
% If not used, comments the lines!
% Requires packages algorithmic, algorithm
\pdfbookmark[1]{List of Algorithms}{loa}
\listofalgorithms
% If Printing on DOUBLE SIDED pages, the second page should be white.
\endgroup
% Otherwise, comment the following command:
\cleardoublepage
% -----------------------------------------------------------------------------
% Listings
% If not used, comments the lines!
% Requires packages listings
\pdfbookmark[1]{Listings}{lol}
\lstlistoflistings
\cleardoublepage
% -----------------------------------------------------------------------------
% % List of acronyms
\pdfbookmark[1]{Acronyms}{loac}
\chapter*{\tlangAcronyms}
% #############################################################################
% This is the ACRONYMS Definition
% !TEX root = ../main.tex
% #############################################################################

\begin{acronym}[H.264/SVC]
	\acro{ODE}{Ordinary Differential Equation}
	\acro{IST}{Instituto Superior Técnico}
\end{acronym}

% If Printing on DOUBLE SIDED pages, the second page should be white.
% Otherwise, comment the following command:
\cleardoublepage
% -----------------------------------------------------------------------------
% PAGE NUMBERING FOR DOCUMENT MATTER in ARABIC
% Pages number is starting with arabic style. Until here were on roman mode
\setcounter{page}{1} \pagenumbering{arabic}
\baselineskip 18pt
% -----------------------------------------------------------------------------
% This a suggestion for the Content of the Document
% Add more Chapters by duplicating a Chapter Block, pointing to the file
%Chapter 1
\acresetall
\fancychapter{Introduction}
\cleardoublepage%
% The following line allows to ref this chapter
\label{chap:intro}%



\section{Motivation}
 
% \par Worldwide, there has been growing interest in the use of autonomous vehicles to execute missions of increasing complexity without constant supervision of human operators.~\cite{xargay2012time} A key-enabling element for the execution of such missions is the availability of advanced methods for cooperative motion planning that take explicitly into account temporal and spatial constraints, intrinsic vehicle limitations and energy minimisation requirements. 
% \par Some autonomous vehicle applications require groups of robots acting cooperatively. An application example that will greatly benefit from vehicle cooperation and that will be focus of this thesis is the control of groups of \acp{AUV} where the visibility is low and obstacles are not known in advance. The \acp{AUV} that together form a robot formation, can, for example, adapt better to unforeseen circumstances in the terrain by making better use of the larger environments that they can observe as the spatial distance between each \ac{AUV} can be varied.
% \par This project has evolved from earlier investigations on underwater mapping connected with the European R\&D Project "MORPH" Project~\cite{morph,aguiar2009cooperative,abreu2016widely} that Insitituto Superior Técnico was a part of. Figure~\ref{fig:morph} illustrates \acp{AUV} in operation on the Morph project.


\par Worldwide, there has been growing interest in the use of autonomous vehicles to execute missions of increasing complexity without constant supervision of human operators.~\cite{xargay2012time} A key-enabling element for the execution of such missions is the availability of advanced methods for cooperative motion planning that take explicitly into account temporal and spatial constraints, intrinsic vehicle limitations and energy minimisation requirements. 
\par Some autonomous vehicle applications require groups of robots acting cooperatively. An application example that will greatly benefit from vehicle cooperation and that will be focus on this thesis is the control of groups of \acp{AUV} where the visibility is low and obstacles are not known in advance. The \acp{AUV} that together form a robot formation, can, for example, adapt better to unforeseen circumstances in the terrain by making better use of the larger environments that they can observe as the spatial distance between each \ac{AUV} can be varied.
\par This project has evolved from earlier investigations on underwater mapping connected with the European R\&D Project "MORPH" Project~\cite{morph,aguiar2009cooperative} and the "WiMUST" Project~\cite{abreu2016widely} that Insitituto Superior Técnico was a part of. Figure~\ref{fig:morph} illustrates \acp{AUV} in operation on the Morph project.
\par The "WiMUST" project, in particular, was composed by a small fleet of \acp{AUV} towing streamers with hydrophones to acquire sub-bottom profiling acoustic data.
\par Recent advancements on the usage of \textit{Bernstein Polynomials} for control also appear to be advantageous,\todo{if mention this, reference that venanzio, isaac pascoal paper}, shows that it's possible to control a high number of vehicles with small computation time.

\begin{figure}[h!]
    \centering
    \includegraphics[width=0.5\textwidth]{Images/projects/Picture2.png}
    \caption{Morph Project}
    \label{fig:morph}
\end{figure}
    
\begin{figure}[h!]
    \centering
    \includegraphics[width=0.6\textwidth]{Images/projects/Picture1.png}
    \caption{Morph Project}
    \label{fig:morph}
\end{figure}


\section{Background}

\par The work discussed in this thesis emphasises in the use of motion planning for multiple cooperative vehicles. Motion Planning, as the name suggests, consists in planning motion for robots, such as mobile vehicles or robotic arms. The choice of control law, i. e., the input which is provided to a robot, will result in a motion which is optimal according to a certain criteria. This criteria could be, for example, minimising time or consumed fuel. Formulisation of such a problem is known as an optimal control problem.
\par There are two main families of techniques for solving optimal control problems: direct methods and indirect methods.
\par Indirect methods 
% https://math.stackexchange.com/questions/946343/optimal-control-difference-between-indirect-direct-approaches
\par Direct methods in optimal control convert the optimal control problem into an optimization problem of a standard form and then using a nonlinear program to solve that optimization problem. 

\par Different models to represent vehicles exist such as a Dubin's car, Medusa Model.
% \par There are two main strategies for motion planning, one is \ac{TT} and the other is \ac{PF}. The first requires a vehicle to follow a path parameterized in space and time, whereas the latter only requires the vehicle to follow a path parameterized in space. Regardless of the strategy, the resulting evirnomental and dynamic constraints are met. Environmental constraints may include inter vehicular constraints, obstacle avoidance. Dynamic constraints include respecting maximum torque, acceleration, velocity and others for each vehicle.



\section{Objectives}


\par There are several ways to obtain a trajectory, for example, single and multiple shooting, collocation and quadratic programming. However, polynomial methods based on Bezier curves are particularly advantageous because they have favourable geometric properties which allow the efficient computation of the minimum distance between trajectories. As the complexity of the polynomials increases, the solutions converge to the optimal.
\par The cost can be constructed based on several criteria such as time and consumed energy. For \textit{cooperative} motion planning, the cost will have to be constructed differently because it will have to take into account the motion of the multiple vehicles at once, in particular, possible inter-vehicle collision.


\begin{itemize}
    \item test some methods for obstacle avoidance
    \item compare some different parameterization methodoligies
    \item analize the complexity of increasing order and number of vehicles
\end{itemize}


\par This 

\section{Overview}

\par A path is a parameterized curve, which is a function that maps a segment $[a,b]$ to $\mathbb{R}^3$. If the parameter of path $p$ is time or a function of time, the map $t\mapsto p(t)$ is called a trajectory.
\par \ac{PF} refers to the problem of making vehicles converge to and follow a path with no explicit temporal schedule while \ac{TT} is the problem of making a vehicle track a trajectory such that both spacial and temporal schedules are satisfied simultaneously.
\par Motion Planning consists in the design of trajectories for different kinds of systems that can later be tracked.
\par A trajectory is the path that an object with mass in motion follows through space as a function of time. Trajectory tracking is the 
\par The objective of motion planning is to find a trajectory to be tracked by a robot in an optimal way, based on a "cost function".
\par 


%\section{Optimization methods: a brief survey}
\section{Problem Statement}



\par Discuss the difference between trajectory tracking and motion planning
\par Discuss the different models for vehicles
\par Maybe discuss optimization software like fmincon (J Garcia does this)
\par Maybe discuss the use of log barrier function
% \subsection{Polynomial methods}
% \subsection{Shooting}
% \section{}



\section{Thesis Outline}


% to use refs: \ref{chap:available_methods} try \Cref{} too. it works for more than just  chapters
\par In chapter \ref{chap:theory}, an overview of the different available numerical methods for the motion planning for a single vehicle will be presented. \todo{exapnd}
\par In chapter \ref{chap:autonomousvehiclemodels}, an overview of different vehicle models is presented. \todo{expand}
\par In chapter \ref{chap:implementation}, a discussion of the code structure is discussed, along with the choice of optimisation algorithms. \todo{exapnd}
\par In chapter \ref{chap:results}, some results are presented. \todo{exapnd}
\par In chapger \ref{chap:conclusion}, the conclusion is made. \todo{exapnd}

This will be followed, in chapter , by application examples for a double integrator in 1 and 2 dimensions that capture the dynamics of a single vehicle. In chapter , some considerations for the control of multiple cooperative vehicles will be presented. The report will be concluded with a final overview of the different methods considered and a plan for the project's work will be defined.

% If Printing on DOUBLE SIDED pages, the second page should be white.
% Otherwise, comment the following command:
\cleardoublepage
%
%Chapter 2
% #############################################################################
% This is Chapter 2
% !TEX root = ../main.tex
% #############################################################################
% Change the Name of the Chapter i the following line
\fancychapter{This is the Second Chapter}
\cleardoublepage
% The following line allows to ref this chapter
\label{chap:back}

Vivamus auctor leo vel dui. Aliquam erat volutpat. Phasellus nibh. Vestibulum ante ipsum primis in faucibus orci luctus et ultrices posuere cubilia Curae; Cras tempor. Morbi egestas, urna non consequat tempus, nunc arcu mollis enim, eu aliquam erat nulla non nibh. Duis consectetuer malesuada velit. Nam ante nulla, interdum vel, tristique ac, condimentum non, tellus. Proin ornare feugiat nisl. Suspendisse dolor nisl, ultrices at, eleifend vel, consequat at, dolor.
% #############################################################################
\section{Traditional Streaming Technologies}
Cras dictum. Maecenas ut turpis. In vitae erat ac orci dignissim eleifend. Nunc quis justo. Sed vel ipsum in purus tincidunt pharetra \cite{MacAulay:2005fk}. Sed pulvinar, felis id consectetuer malesuada, enim nisl mattis elit, a facilisis tortor nibh quis leo. Sed augue lacus, pretium vitae, molestie eget, rhoncus quis, elit \cite{Schwarz:2007lr}. Donec in augue. Fusce orci wisi, ornare id, mollis vel, lacinia vel, massa. Pellentesque habitant morbi tristique senectus et netus et malesuada fames ac turpis egestas..

Sed pulvinar, \enquote{felis id consectetuer} malesuada, enim nisl mattis elit, a facilisis tortor nibh quis leo \Cref{tab:streamingtech}.

\begin{table}[htb]
\centering
\normalsize
{\footnotesize
    \caption{Streaming Technologies Comparison}
    \label{tab:streamingtech}
    \begin{tabular}{ | c | c | c | c |}
    \hline
    & Dynamic & Smooth & HLS\\
    & Streaming & Streaming & \\ \hline \hline

    Streaming Protocol & RTMP & HTTP & HTTP \\
    %\textbf{Protocol} & & & \\ 
    \hline
    
    Video Codec & H.264, VP6 & H.264 & H.264 \\ 
    %\textbf{Codec} & &  & \\ 
    \hline
    
    Audio Codec & AAC, MP3 & WMA, AAC & AAC, MP3  \\
    %\textbf{Codec} & & & \\ 
    \hline
    
    Container Format & MP4, FLV, & MP4 & MPEG2-TS \\
    %\textbf{Format} & F4V & & \\ 
    \hline
    
     iOS & NO & YES & YES \\ \hline
     
    Android & NO & YES & YES \\ \hline
    
    \end{tabular}
    }
\end{table} 

Suspendisse vestibulum dignissim quam. Integer vel augue. Phasellus nulla purus, interdum ac, venenatis non, varius rutrum, leo. Pellentesque habitant morbi tristique senectus et netus et malesuada fames ac turpis egestas \cite{RFC-VP8}. Duis a eros. Class aptent taciti sociosqu ad litora torquent per conubia nostra, per inceptos hymenaeos. Fusce magna mi, porttitor quis, convallis eget, sodales ac, urna \cite{Chiang:2011fk}. \textcolor{violet}{\Cref{tab:spreadtb} illustrates the use of a Spreadsheet-like table producing calculations by columns and by lines (observe the code).} 

\begin{table}[htb]
\centering
    \caption{A nice Spreadsheet using package ``spreadtab''. Notice the calculations.}
    \label{tab:spreadtb}
\begin{spreadtab}{{tabular}{rr|r}} 
22       & 54       & a1+b1 \\
43       & 65       & a2+b2 \\ 
49       & 37       & a3+b3 \\
\hline
a1+a2+a3 & b1+b2+b3 & a4+b4
\end{spreadtab}
\end{table} 
% #############################################################################
\section{Cras lobortis tempor velit}
Nunc tincidunt convallis tortor. Duis eros mi, dictum vel, fringilla sit amet, fermentum id, sem. Phasellus nunc enim, faucibus ut, laoreet in, consequat id, metus. Vivamus dignissim \cite{Moscoso:2011fk}. \textcolor{violet}{\Cref{tab:comp_arch} is automatically compressed to fit text width. You can use \url{https://www.tablesgenerator.com} to produce these tables, and then copy the \LaTeX\space code generated to paste in the document.}


\begin{table}[h]
\centering
\caption{Comparison between today's and target Architectures of Telcos}
\label{tab:comp_arch}
\resizebox{\textwidth}{!}{%
\begin{tabular}{|
>{\columncolor[HTML]{ECF4FF}}l |l|
>{\columncolor[HTML]{E2FFC9}}l |l|}
\hline
\multicolumn{2}{|c|}{\cellcolor[HTML]{ECF4FF}Today}                                                                                                             & \multicolumn{2}{c|}{\cellcolor[HTML]{E2FFC9}Target}                                                                                  \\ \hline
Rigid     & \begin{tabular}[c]{@{}l@{}}Each evolutionary requirement involves \\ development of multiple components, \\ interfaces, platforms,etc.\end{tabular} & Flexible       & \begin{tabular}[c]{@{}l@{}}It is possible to modify or add new\\ functionalities rapidly.\end{tabular}              \\ \hline
Slow      & \begin{tabular}[c]{@{}l@{}}Development of a new application takes \\ months or years.\end{tabular}                                                  & Fast           & \begin{tabular}[c]{@{}l@{}}Development of a new application takes \\ weeks instead of months or years.\end{tabular} \\ \hline
Closed    & \begin{tabular}[c]{@{}l@{}}Limited integration with external\\ environments.\end{tabular}                                                           & Open           & \begin{tabular}[c]{@{}l@{}}It is simple to integrate internal,\\ applications with external entities.\end{tabular}  \\ \hline
Complex   & \begin{tabular}[c]{@{}l@{}}Heterogeneous technologies, obsolescence, \\ lack,of standards, high redundancy.\end{tabular}                            & Standardised   & Use of homogeneous architectural models.                                                                            \\ \hline
Expensive & \begin{tabular}[c]{@{}l@{}}High Capex (for new service development) \\ and,high,Opex (to ensure running of IT).\end{tabular}                        & Cost-Effective & Capex and Opex are optimised.                                                                                       \\ \hline
\end{tabular}
}
\end{table}

Cras lobortis tempor velit. Phasellus nec diam ac nisl lacinia tristique. Nullam nec metus id mi dictum dignissim. Nullam quis wisi non sem lobortis condimentum. Phasellus pulvinar, nulla non aliquam eleifend, tortor wisi scelerisque felis, in sollicitudin arcu ante lacinia leo.
% If Printing on DOUBLE SIDED pages, the second page should be white.
% Otherwise, comment the following command:
\cleardoublepage
%
%Chapter 3
% #############################################################################
% This is Chapter 3
% !TEX root = ../main.tex
% #############################################################################
% Change the Name of the Chapter i the following line
\fancychapter{This is the Third Chapter}
\cleardoublepage
% The following line allows to ref this chapter
\label{chap:architecture}

Donec gravida posuere arcu. Nulla facilisi. Phasellus imperdiet. Vestibulum at metus. Integer euismod. Nullam placerat rhoncus sapien. Ut euismod. Praesent libero. Morbi pellentesque libero sit amet ante. Maecenas tellus. Maecenas erat. Pellentesque habitant morbi tristique senectus et netus et malesuada fames ac turpis egestas.
% #############################################################################
\section{Architecture Design Requirements} 
\textcolor{violet}{Example of a Flowchart for a system, in \Cref{fig:flowchart}, created with \url{https://www.draw.io} and then exported as ``PDF'' crop format (a true vector image that can be scaled to no end, with no pixels or distortion).}

\begin{figure}[h]
\centering
\includegraphics[width=1.0\textwidth]{./Images/Flowchart_from_draw-io.pdf}
\caption{System Processes}
\label{fig:flowchart}
\end{figure}

Quisque facilisis erat a dui. Nam malesuada ornare dolor. Cras gravida, diam sit amet rhoncus ornare, erat elit consectetuer erat, id egestas pede nibh eget odio. Proin tincidunt, velit vel porta elementum, magna diam molestie sapien, non aliquet massa pede eu diam. Aliquam iaculis. Fusce et ipsum et nulla tristique facilisis. Donec eget sem sit amet ligula viverra gravida. Etiam vehicula urna vel turpis. 

\textcolor{violet}{And here another diagram of a network (\Cref{fig:network}) created with \url{https://www.draw.io} and then exported as ``PDF'' crop format.}

\begin{figure}[h]
\centering
\includegraphics[width=1.0\textwidth]{./Images/Network_from_draw-io.pdf}
\caption{Network Diagram}
\label{fig:network}
\end{figure}

Suspendisse sagittis ante a urna. Morbi a est quis orci consequat rutrum. Nullam egestas feugiat felis. Integer adipiscing semper ligula. Nunc molestie, nisl sit amet cursus convallis, sapien lectus pretium metus, vitae pretium enim wisi id lectus. Donec vestibulum. Etiam vel nibh. Nulla facilisi. Mauris pharetra. Donec augue. Fusce ultrices, neque id dignissim ultrices, tellus mauris dictum elit, vel lacinia enim metus eu nunc:

\begin{description}
	\item[\textbf{Web-streaming:}]
	The client application should support streaming media using \ac{HTTP} protocols.
	\item[\textbf{Multi-source streaming:}]
	The client application should support multi-source streaming media, i.e., ``simultaneous'' streaming of media content components from a network, supported\slash complemented by \ac{CDN}\slash \ac{CC} services. 
	\item[\textbf{Support content Metadata Description:}]
	The client application should support content metadata description in a format similar or compliant with MPEG \ac{DASH} \cite{ISO/IEC:2012fk}. 
	\item[\textbf{Scalable and Adaptive Media Contents:}]
	The system should support on-demand streaming of scalable and adaptive contents based on \ac{SVC}.
	\item[\textbf{Heterogenous End-User Devices:}]
	The client application should be compatible with current and future generations of end-user devices form factors, irrespective of their performance, screen size and resolution.
	\item[\textbf{Access Network independency:}] 
	The solution should provide the expected service over different types of access networks supported by the end-user devices, such as Wireless \acp{LAN} (IEEE 802.11) or cellular data networks such as \ac{GPRS}, \ac{UMTS}, \ac{LTE}, etc.
\end{description}

Cras gravida, diam sit amet rhoncus ornare, erat elit consectetuer erat, id egestas pede nibh eget odio. Proin tincidunt, velit vel porta elementum, magna diam molestie sapien, non aliquet massa pede eu diam. Aliquam iaculis. Fusce et ipsum et nulla tristique facilisis.
% #############################################################################
\section{Architecture Design Requirements}
Ut nulla. Vivamus bibendum, nulla ut congue fringilla, lorem ipsum ultricies risus, ut rutrum velit tortor vel purus. In hac habitasse platea dictumst. Duis fermentum, metus sed congue gravida, arcu dui ornare urna, ut imperdiet enim odio dignissim ipsum. Nulla facilisi. Cras magna ante, bibendum sit amet, porta vitae, laoreet ut, justo. Nam tortor sapien, pulvinar nec, malesuada in, ultrices in, tortor. Cras ultricies placerat eros. Quisque odio eros, feugiat non, iaculis nec, lobortis sed, arcu. Pellentesque sit amet sem et purus pretium consectetuer \Cref{mpd}\todo[color=cyan!40, author=RC, fancyline]{A listing for XML code, with syntax highlighting}{}.

\begin{minipage}[c]{0.95\textwidth}
\begin{center}
\begin{spacing}{0.5}
\begin{lstlisting}[frame=lines,style=XML,caption={Example of a MPD file.},label=mpd]
<?xml version="1.0" encoding="UTF-8"?>
<StreamInfo version="2.0">
    <Clip duration="PT01M0.00S">
        <BaseURL>videos/</BaseURL>
        <Description>svc_1</Description>
        <Representation mimeType="video/SVC" codecs="svc" frameRate="30.00" bandwidth="401.90"
            width="176" height="144" id="L0">
            <BaseURL>svc_1/</BaseURL>
            <SegmentInfo from="0" to="11" duration="PT5.00S">
                <BaseURL>svc_1-L0-</BaseURL>
            </SegmentInfo>
        </Representation>
        <Representation mimeType="video/SVC" codecs="svc" frameRate="30.00" bandwidth="1322.60"
            width="352" height="288" id="L1">
            <BaseURL>svc_1/</BaseURL>
            <SegmentInfo from="0" to="11" duration="PT5.00S">
                <BaseURL>svc_1-L1-</BaseURL>
            </SegmentInfo>
        </Representation>
    </Clip>
</StreamInfo>
\end{lstlisting}
\end{spacing}
\end{center}
\end{minipage}

Nam malesuada ornare dolor. Cras gravida, diam sit amet rhoncus ornare, erat elit consectetuer erat, id egestas pede nibh eget odio. Proin tincidunt, velit vel porta elementum, magna diam molestie sapien, non aliquet massa pede eu diam.
% If Printing on DOUBLE SIDED pages, the second page should be white.
% Otherwise, comment the following command:
\cleardoublepage
%
%Chapter 4
% #############################################################################
% This is Chapter 4
% !TEX root = ../main.tex
% #############################################################################
% Change the Name of the Chapter i the following line
\fancychapter{Implementation}
\cleardoublepage
% The following line allows to ref this chapter
\label{chap:implementation}

\par In this chapter, implementation in both Matlab and Python will be discussed. This chapter focuses on formalising the optimisation problem for multipe vehicles. Once the optimisation problem is formalised, it can be solved by a whole range of non linear optimisation algorithms.

\par The Motion Planning problem for multiple vehicles that will be focused for the project consists in a go-to formation manoeuvre \cite{sabetghadam2018cooperative}. The go-to formation manoeuvre consists of the simultaneous arrival of a formation of vehicles to desired locations whilst simultaneously avoiding collisions between each other and the environment.

\par Just as for the case of optimisation for a single vehicle, motion planning for multiple vehicles will also be based on the minimisation of an appropriate cost function. This time, however, the cost will be the result of the sum of the costs of each individual vehicle and an added constraint will be necessary that will take into account inter vehicle collisions. 

\par The optimal control problem can be redefined for multiple vehicles as 
\begin{equation}
    \label{eq:multi_cost}
    \begin{aligned}
    & \underset{x^{[i]}(.),u^{[i]}(.),i= 1,\dots N_v}{\text{minimize}} && \int_0^T \sum_{i=1}^{N_v}  L_i (x^{[i]}(t),u^{[i]}(t))dt + \Psi (x(T)) \\
    & \text{subject to}  && x^{[i]}(0) = x_0^{[i]}, \\
        & && x^{[i]}(T) = x_f^{[i]}, \\
        & && \dot{x}^{[i]} = f_i (x^{[i]}(t), u^{[i]}(t)), &&& t \in [0,T]\\
        & && c_{col} (x^{[i]},u^{[i]} ) \geq 0, \\
        & && h(x(t),u(t)) \geq 0, \\
        & && \underline{x}^{[i]} \leq x^{[i]} (t) \leq \overline{x}^{[i]} , \\
        & && \underline{u}^{[i]} \leq u^{[i]} (t) \leq \overline{u}^{[i]}
    \end{aligned}
\end{equation}

\par This problem will also produce a solution within time horizon $T$. Initial and terminal constraints will exist for every vehicle, as well as inter-vehicle constraints.

\par A problem that only has the integral term $\sum_{v=1}^{N_v} L_v(x^{(v)}(t),u^{(v)}(t))$ is said to be in \textit{Lagrange form}, a problem that optimises only the boundary objective $\sum_{v=1}^{N_v} \Psi_v(x(T))$ is said to be in \textit{Mayer form} and a problem with both terms is said to be in \textit{Bolza form}. An example where only the \text{Mayer form} would be necessary could be a situation where the desired destination of the vehicles does not have enough room for them all to be arranged in their desired positions. Therefore, the goal of the optimiser is to find the closest to the desired positions.


\par There are 2 ways of preventing inter-vehicle collision; \textit{spatial deconfliction} and \textit{temporal deconfliction} \cite{hausler2015mission}.
Spatial deconfliction imposes the constraint that the spatial paths of the vehicles under consideration will never intersect and keep a desired safe distance from each other. Temporal deconfliction requires that two vehicles will never be ”at the same place at the same time”. However, their spatial paths are allowed to intersect. Figure \ref{fig:deconfliction} illustrates the two types of deconfliction strategies. Temporal deconfliction allows an extra degree of freedom and will intuitively lead to cheaper dynamic costs.

\par A simple method to assure temporal deconfliction of 2 vehicles is to assure that the norm of the distance of each point in time $t\in [0,T]$ is on each point in time subtract the n-dimensional trajectories of each pair of vehicle 

\begin{figure}
    \centering
    \begin{subfigure}[b]{0.45\textwidth}
        \includegraphics[width=\textwidth]{Images/spacial_deconf.jpg}
        \caption{Spacial Deconfliction}
    \end{subfigure}
    ~
    \begin{subfigure}[b]{0.45\textwidth}
        \includegraphics[width=\textwidth]{Images/temporal_deconf.jpg}
        \caption{Temporal Deconfliction}
    \end{subfigure}
    \caption{Inter Vehicle deconfliction Solutions}
    \label{fig:deconfliction}
\end{figure}

\section{Description of the implemented code}

\par The variables for the motion planning problem are each vehicle's state variables and inputs, as explained in chapter \label{chap:autonomousvehiclemodels}. Each of these variables will be refered to as curves. Optimisation algorithms cannot take continuous functions as variables, therefor, some form of parameterisation of each curve is necessary, as exemplified in some of the algorithms of chapter \label{chap:theory}. Here, each curve will be represetend as a Bernstein Polynomial with order $N$, which will require $N+1$ control points. A destinction is made between state variables and inputs: state variables must have establised initial and final conditions, inputs do not. This implies that for state variables, the initial and final control points must be fixed, therefor, they do not need to participate on the optimisation problem. The following matrix represents how the control points are stored so that they can be accessed by all functions that perfom operations on the curves

\begin{equation}
    \begin{bmatrix}
        \colorbox{yellow}{$\displaystyle x_0^0$} & \colorbox{yellow}{$\displaystyle y_0^0$} & \colorbox{yellow}{$\displaystyle \psi_0^0$} & \colorbox{yellow}{$\displaystyle u_0^0$} & \colorbox{yellow}{$\displaystyle v_0^0$} & \colorbox{yellow}{$\displaystyle r_0^0$} & \tau_{u_0}^0 & \tau_{r_0}^0 \\
        x_1^0 & y_1^0 & \psi_1^0 & u_1^0 & v_1^0 & r_1^0 & \tau_{u_1}^0 & \tau_{r_1}^0 \\
        \vdots & \vdots & \vdots & \vdots & \vdots & \vdots & \vdots & \vdots \\
        \colorbox{yellow}{$x_{N}^0$} & \colorbox{yellow}{$y_{N}^0$} & \colorbox{yellow}{$\psi_{N}^0$} & \colorbox{yellow}{$u_{N}^0$} & \colorbox{yellow}{$v_{N}^0$} & \colorbox{yellow}{$r_{N}^0$} & \tau_{u_{N}}^0 & \tau_{r_{N}}^0
    \end{bmatrix}
    \label{eq:matrixofvariables}
\end{equation}

\par It is believed \todo{can this be said? because I'm not referencing anything (I'm taking Venanzio's word for it)} that \texttt{fmincon()} performs better when all variables are stored in a line vector, therefor, a function called \texttt{matrify()} is necessary in order to transform the flattened optimisation variable to the matrix of equation (\ref{eq:matrixofvariables}). 
\par Elements marked in yellow do not participate in the optimisation algorithm. They are concatenated to this matrix in \texttt{matrify()}. 

\par High orders are preferable for each curve because \todo{reference "the paper" (of Venanzion, Isaac, Pascoal, etc) } the higher the curve, the closer the control points are to the "matching" point in time of the curve, which is achieved once an optimisation problem finishes. Chapter \ref{chap:results} exemplifies show the control points approximate to the curve and how it is advantageous to produce the dynamics.

\par A more precise way to calculate the minimum distance of a Bezier curve to a point or to another curve, compared to the one implemented on section \ref{sec:2_d_bezier} is via the algorithm presented in \cite{chang2011computation}. This algorithm takes into account the convex hull property of Bezier Curves and the \textit{deCasteljau} algorithm for subdividing curves. It will also employ the GJK algorithm, a fast and efficient way of calculating distances between 2 convex shapes\cite{cichella2018bernstein}.
\par The fast calculation of distances between these curves allows Bezeir Curves to be appropriate for testing non-linear constraints in multiple vehicle motion planning optimisation.

\par This fancy algorithm performed poorly because it was iterative. A simpler way to calculate the minimum distance was necessary. This consited in subtracting each pair of 2-D curves for each vehicle to eachother, then getting a rough guess of the closest point of that resulting curve to 0: by performing degree elevation to an order 10 times the original then looking for the closest point to the origin. 

\par Minimum distance of the curves to objects was done with GJK. but it was slow

\par Collision to circles is calculated by subtracting the 2-D curve tto the centre of the circle, performing degree elevation of the resultign curve, calculating the closest to the origin and checking wether that distance is greater to the radius. Performing the check for every control point was also done but wasn't advantageous (slower runtime), the reason could be because the derivative of each of these distances or the elevated curve with respect to the position of the control points depends on more than 1 control points of the non elevated curve, therefor, some computation is redundant.


\par The optimisation problem is formulated by constructing a data structure with the fields of table \ref{tab:constants_description}. In Python the same fields are necessary, however, they must be stored in a python dictonary. 


\begin{table}[]
\centering
\begin{tabular}{|l|l|l|l|}
\hline
\textbf{field} & \textbf{description} & \textbf{mandatory} & \textbf{example} \\ \hline
\texttt{T} & Time horizon & yes & \texttt{10} \\ \hline
\texttt{xi} & initial conditions & yes & \texttt{[0 0 0 1 0]} \\ \hline
\texttt{xf} & final conditions & yes & \texttt{[5 5 pi/2 1 0]} \\ \hline
\texttt{N} & order of the curves & yes & \texttt{15} \\ \hline
\texttt{obstacles} & polygons & \begin{tabular}[c]{@{}l@{}}no\\ default: \texttt{[]}\end{tabular} &  \\ \hline
\texttt{obstacles\_circles} & circles & \begin{tabular}[c]{@{}l@{}}no\\ default: \texttt{[]}\end{tabular} &  \\ \hline
\texttt{min\_dist\_int\_veh} & \begin{tabular}[c]{@{}l@{}}minimum distance between \\ vehicles for every point in time\end{tabular} & \begin{tabular}[c]{@{}l@{}}no\\ default: \texttt{0}\end{tabular} & \texttt{.8} \\ \hline
\texttt{numinputs} & \begin{tabular}[c]{@{}l@{}}number of input variables\\ (don't have initial conditions)\end{tabular} & \begin{tabular}[c]{@{}l@{}}no\\ default: \texttt{0}\end{tabular} &  \\ \hline
\texttt{uselogbar} & \begin{tabular}[c]{@{}l@{}}make the problem completely \\ unconstrained and use log \\ barrier functionals\end{tabular} & \begin{tabular}[c]{@{}l@{}}no\\ default: \texttt{false}\end{tabular} &  \\ \hline
\texttt{usesigma} & \begin{tabular}[c]{@{}l@{}}a boolean for the usage of the \\ sigma function if log barrier \\ functionals are to be used\end{tabular} & \begin{tabular}[c]{@{}l@{}}no\\ default: \texttt{false}\end{tabular} &  \\ \hline
\texttt{costfun\_single} & \begin{tabular}[c]{@{}l@{}}a function used to calculate \\ the running cost for each \\ singular vehicle\end{tabular} & yes & \texttt{@costfun} \\ \hline
\texttt{dynamics} & \begin{tabular}[c]{@{}l@{}}a function that describes how the \\ non linear dynamics of the state \\ variables and inputs are linked\end{tabular} & yes & \texttt{@dynamics} \\ \hline
\texttt{init\_guess} & \begin{tabular}[c]{@{}l@{}}a function that provides an initial \\ guess for the optimisation problem \\ which may  speed up the process \\ of optimisation\end{tabular} & \begin{tabular}[c]{@{}l@{}}no \\ default:\\ \texttt{@rand\_init\_guess}\end{tabular} & \texttt{@init\_guess} \\ \hline
\texttt{recoverxy} & \begin{tabular}[c]{@{}l@{}}a function that returns the x and y\\ variables by solving just the initial \\ value problem of the inputs\end{tabular} & yes & \texttt{@recoverxy} \\ \hline
\end{tabular}
\caption{Description of the constants for optimisation}
\label{tab:constants_description}
\end{table}


\par Some notes for each of the fields:

\begin{itemize}
    \item \texttt{xi} has as many lines as state variables (not input variables) and as many lines as number of vehicles. Because these functions are designed for vehicles, $x$ and $y$ must be in the first 2 columns
    \item \texttt{xf} works just as \texttt{xi}
    \item \texttt{obstacles\_circles}  Ncircles by 3, where columns are x, y and radius, respectively
    \item \texttt{recoverxy} takes an aribtrary $X$ matrix and and a \texttt{constants} structure and returns a Npoints by 2 matrix
    \item \texttt{dynamics} takes in an arbitrary X matrix and constants structure (to provide pre computer information like a derivation matrix) and must return a column vector which is zeros when all of the dynamic constraints are respected
\end{itemize}

\par The data structure for the nonlinear optimisation problem is then passed to 

The running cost function will be based on minimisation of the energy:
\begin{equation}
    J = \int tau_u^2 + tau_r^2
\end{equation}

\subsection{dynamics}

the dynamics in matlab is 

\begin{lstlisting}[language=matlabfloz,caption={\mcode{Matlab Function}}]
ceq = [
    DiffMat*x - u.*cos(yaw) + v.*sin(yaw) - Vcx;
    DiffMat*y - u.*sin(yaw) - v.*cos(yaw) - Vcy;
    DiffMat*yaw - r;
    DiffMat*u - 1/m_u*(tau_u + m_v*v.*r - d_u.*u+fu);
    DiffMat*v - 1/m_v*(-m_u*u.*r - d_v.*v+fv);
    DiffMat*r - 1/m_r*(tau_r + m_uv*u.*v - d_r.*r+fr);
];
\end{lstlisting}

\begin{itemize}
    \item diffmat preserves the order
    \item the equailty is maintained in the control points, not the values of the curve itself so some error is expected
\end{itemize}
diff ma
\todo{explain the ordinary differential equaitons}

% If Printing on DOUBLE SIDED pages, the second page should be white.
% Otherwise, comment the following command:
\cleardoublepage
%
%Chapter 5
% #############################################################################
% This is Chapter 5
% !TEX root = ../main.tex
% #############################################################################
% Change the Name of the Chapter i the following line
\fancychapter{Results}
\cleardoublepage
% The following line allows to ref this chapter
\label{chap:results}

\par The following results are based on solving the optimisation problem \eqref{eq:multi_cost_bern} with the implementation described in section \ref{sec:description_implementation}.
\par Sequential Quadratic Programming \cite{10.1007/978-0-387-35514-6_7} will be the nonlinear programming solver of choice. Simulations were run on a 4 × Intel\textsuperscript{\textcopyright} Core\texttrademark i5-7200U CPU @ 2.50GHz processor. 


\par Several different running cost functions were tested such as 
\begin{equation}
    J = \int_0^T \frac{du}{dt}^2 dt
\end{equation}
which minimizes tangent acceleration,
\begin{equation}
    J = \int_0^T u^2 dt
\end{equation}
which minimizes speed, and finally, for the Medusa model, specifically,
\begin{equation}
    J = \int_0^T \tau_u^2 + \tau_r^2
\end{equation}
which minimizes the input.
\par All of these serve as proxies to the minimisation of spent energy.

\par Results for the two models presented in chapter \ref{chap:autonomousvehiclemodels} are presented. The unicycle model has a total of 5 state variables while the Medusa model has a total of 6 state variables plus 2 inputs. Upper and lower bounds for each variable for each vehicle were implemented as explained in section \ref{sec:dynamics}, by finding the biggest and smallest control points. For the examples presented in this chapter, the bounds that were used are those presented in table \ref{tab:variablebounds} which were chosen to closer represent a real Medusa vehicle.
\begin{table}[h!]
\centering
\begin{tabular}{|l|l|l|l|}
\hline
& Variable & Starting Conditions & Final Conditions \\ \hline
Dubin's Car & $x$ & $-\infty$ & $\infty$ \\
& $y$ & $-\infty$ & $\infty$ \\
& $\psi$ & $-\infty$ & $\infty$ \\
& $u$ & 0 & 1.1 \\
& $r$ & $-\pi/4$ & $\pi/4$ \\ \hline
Medusa & $x$ & $-\infty$ & $\infty$ \\
& $y$ & $-\infty$ & $\infty$ \\
& $\psi$ & $-\infty$ & $\infty$ \\
& $u$ & 0 & 1.1 \\
& $v$ & $-\infty$ & $\infty$ \\
& $r$ & $-.74$ & $.74$ \\
& $\tau_u$ & 0 & 25.9 \\
& $\tau_r$ & -.113 & .113 \\
\hline
\end{tabular}
\caption{Upper and lower bounds for each variable of each vehicle model}
\label{tab:variablebounds}
\end{table}

\par A significant number of experiments were performed to the optimisation algorithm in order to study its behaviour with changing parameters. Out of all of the experiments that were performed, the most relevant will be presented.

\section{No obstacles}

\par The first problem will be run for both a single Dubin's car and a single Medusa vehicle, with initial and final states described in table \ref{tab:firstproblem} and no obstacles. Figures \ref{fig:noobstaclesfigures} and \ref{fig:noobstaclesmedusa} show solutions for order 20 and a time horizon of . Each example has a different combination of vehicle model and cost function. Both models contain control points to describe $x$ and $y$ positions, which is what the blue lines show. The red lines describe the solution of the Initial Value Problem as described in section \ref{sec:ivproblem}. This figure, and all that remain, flip x and y axis which standard for marine vehicles.

\begin{table}[h!]
\centering
\begin{tabular}{|l|l|l|}
\hline
Variable & Starting Conditions & Final Conditions \\ \hline
$x$ & 0 & 30 \\
$y$ & 0 & 30 \\
$\psi$ & 0 & $\pi/2$ \\
$u$ & 1 & 1 \\
$v$ & 0 & 0 \\
$\omega$ & 0 & 0 \\
\hline
\end{tabular}
\caption{Initial and final conditions for a basic Motion Problem}
\label{tab:firstproblem}
\end{table}


\begin{figure}[h!]
\centering
\includegraphics[width=0.8\textwidth]{Images/results/noostaclesfigures.png}
\caption{Solutions of order $N=20$ without obstacles and final time $T=60$}
\label{fig:noobstaclesfigures}
\end{figure}

\begin{figure}[h!]
\centering
\includegraphics[width=0.8\textwidth]{Images/results/noostaclesmedusa.png}
\caption{Solutions of order $N=20$ that minimize $\tau_u^2$ (left) and $\tau_r^2$ (right) for the Medusa Model with time horizon $T=60$}
\label{fig:noobstaclesmedusa}
\end{figure}

\par First thing to note is, despite all executions running successfully, i. e., the optimal and feasible solution was found, the Medusa's \ac{IVP} solution resembles less the plot of the Bezier curve of the x and y control points when compared with the Dubin's car. This suggests that the order isn't high enough for the curves to accurately represent the real optimal solution's state variables for the Medusa model, which is more complex. 


\section{Obstacles}

\par The following figures show solutions with circle or polygon obstacles whose collision avoidance algorithms are explained in sections \ref{sec:mindistintveh} and \ref{sec:mindistconvshapes}. Figure \ref{fig:baselineresult} is the baseline example without obstacles. It uses a Dubin's Car with the same initial and final conditions as in the previous section and minimises $v^2$. Figure \ref{fig:circleobstacle}, shows the solution with the added circle as a constraint. Figure \ref{fig:circleobstaclelogbargood} shows the solution with an obstacle but the constraint was moved to the log barrier functional as explained in section \ref{sec:logbarrierfunc}. Figure \ref{fig:circleobstaclelogbarbad} shows the solution with log barrier funcitonal as well but the relative weight of the log barrier on the cost wasn't as big, and, as a result, the optimisation problem terminated successfully but did not prevent collision. The runtimes of these examples don't show how the usage of log barrier provides an advantage, however, for a polygon obstacle, such as in figures \ref{fig:polygonobstacle} and \ref{fig:polygonobstaclelogbar} show a huge difference in the usage of the log barrier functional. Both these results have a huge runtime when compared to circular obstacles because the algorithm is iterative.

\begin{figure}[h!]
\centering
\includegraphics[width=0.8\textwidth]{Images/results/baselineresult.png}
\caption{Solution of order $N=20$ without obstacles \\ computation time = 2s}
\label{fig:baselineresult}
\end{figure}

\begin{figure}[h!]
\centering
\includegraphics[width=0.8\textwidth]{Images/results/circleobstacle.png}
\caption{Circle osbtacle: computation time 5s}
\label{fig:circleobstacle}
\end{figure}

\begin{figure}[h!]
\centering
\includegraphics[width=0.8\textwidth]{Images/results/circleobstaclelogbargood.png}
\caption{Circle obstacle plus log bar: computation time 4s}
\label{fig:circleobstaclelogbargood}
\end{figure}

\begin{figure}[h!]
\centering
\includegraphics[width=0.8\textwidth]{Images/results/circleobstaclelogbarbad.png}
\caption{Circle obstacle plus log bar: computation time 3s}
\label{fig:circleobstaclelogbarbad}
\end{figure}

\begin{figure}[h!]
\centering
\includegraphics[width=0.8\textwidth]{Images/results/polygonobstacle.png}
\caption{polygon obstacle: computation time 307s}
\label{fig:polygonobstacle}
\end{figure}

\begin{figure}[h!]
\centering
\includegraphics[width=0.8\textwidth]{Images/results/polygonobstaclelogbar.png}
\caption{polygon obstacle: computation time 157s}
\label{fig:polygonobstaclelogbar}
\end{figure}


\subsection{Variation of cost with order}

\par The next step was to implement a way of calculating solutions for high orders while maintaining low computation time. This is acheived by taking the solution of a low order, perform degree elevation and re-feed that solution as initial guess for a higher order. Figure \ref{fig:progressiveNexamples}, show some solutions of this procress. The top left figure started is the solution order 10, this solution has it's order increased by 10, and used as initial guess for another run resulting in the top right figure and so on and so on. The iterative process stopped with order 70 because the relative final cost differs from order 60 by less than 1\%. The same solution of oder 70 took a total of 334 seconds when using a random initial guess which shows how this iterative method can save computation time.

\begin{figure}[h!]
\centering
\includegraphics[width=0.8\textwidth]{Images/results/progressiveNexamples.png}
\caption{Results of Interatively increasing order}
\label{fig:progressiveNexamples}
\end{figure}



\par We can see how to cost varies with increase of order $N$ and how the solution of the \ac{IVP} becomes closer and closer to the plot of $xy$.





\section{Multiple Vehicles}

\par In figure \ref{fig:multiplevehicles} we can see how computation time quickly grows with even a small number of vehicles for a Medusa model and using the sampling approach for deconfliction discussed in \ref{sec:mindistintveh}.

\begin{figure}[h!]
\centering
\includegraphics[width=0.8\textwidth]{Images/results/multiplevehicles.png}
\caption{Solutions of order $N=10$ with multiple vehicles}
\label{fig:multiplevehicles}
\end{figure}


\par And at last an example with 3 vehicles and 1 obstacle, in figure \ref{fig:finalexample}

\begin{figure}[h!]
\centering
\includegraphics[width=0.8\textwidth]{Images/results/finalexample.png}
\caption{Solution of order $N=10$ with 3 vehicles and 1 circle obstacle}
\label{fig:finalexample}
\end{figure}
% If Printing on DOUBLE SIDED pages, the second page should be white.
% Otherwise, comment the following command:
\cleardoublepage
%
%Chapter 6
% #############################################################################
% This is Chapter 6
% !TEX root = ../main.tex
% #############################################################################
% Change the Name of the Chapter i the following line
\fancychapter{Conclusion}
\cleardoublepage%
\label{chap:conclusion}

% #############################################################################

\par In this work, a fast optimal motion planning algorithm was designed. It consisted in solving an optimal control problem, by finding its equivalent optimisation problem. The algorithm's final form was chosen based on the comparison of several parameterisation methods and how well each one can handle dynamic and environmental constraints. Bezier curves was the final choice of parameterisation to approximate the optimal trajectory. They have very good properties for motion planning that greatly simplify the optimisation problem.

Two \ac{AUV} models, the unicycle and the Medusa, were studied and afterwards, how can their characteristics reap the benefits from the proposed motion planning algorithm. Specifically, it is shown that the use of a Bernstein based polynomial representation of trajectories is no longer limited for differentially flat systems and how, in fact, defining all of the state variables and inputs and linking them via the dynamics simplifies the computation of running costs and calculating feasibility of the solution.
\par Results also show, however, that time complexity quickly increases with a high number of vehicles, specially when a high order of approximation is used for every vehicle's state and input variables. These time complexity limitations could potentially be negligible if a more advanced computational power is available.
\par This algorithm, not only solves the problem for the go-to-formation maneuver, to start cooperative missions, but also allows to add other objectives such active navigation localization. The tools developed to plan optimal trajectories can now be explored to conduct further research. 


% If Printing on DOUBLE SIDED pages, the second page should be white.
% Otherwise, comment the following command:
\cleardoublepage
%
% -----------------------------------------------------------------------------
% BIBLIOGRAPHY
% Add the Bibliography to the PDF table of contents (not the document table of contents)
\pdfbookmark[0]{Bibliography}{bib}
% The bibliography style sheet
% Chose your preferences on the format of the entries and the Labels:
% IEEEtran: Used in general (recommended for IST Thesis)
%           Entries are labelled and sorted by appearance in the document
%           Labels are Numeric inside square brackets
\bibliographystyle{IEEEtran}
%
% Apalike:  Entries formatted alphabetically, last name first, with identation
%           Labels with Autor's Name and Year inside square brackets
%\bibliographystyle{apalike}
%
% Alpha:    Entries formatted with Autor's Name and Year, hanging identation
%           Labels with Autor's abbr. Names and Year inside square brackets
%\bibliographystyle{alpha}
%
% Acm:     Entries formatted with Autor's Name (small Caps), hanging identation
%          Labels are Numeric inside square brackets
%\bibliographystyle{acm}
% The following command resets the 'emphasis' style for bibliography entries
\normalem
% Name of your BiBTeX file
\bibliography{./Thesis-MSc-Bibliography} % Put here your own filename
%
% The following command modifies the 'emphasis' style for bibliography entries
\ULforem
% If Printing on DOUBLE SIDED pages, the second page should be white.
% Otherwise, comment the following command:
\cleardoublepage
%
% -----------------------------------------------------------------------------
% HERE GO THE APPENDIXES IF REQUIRED
% If not required just comment the blocks
\appendix
%% First Appendix
\pdfbookmark[1]{Appendix A}{appendix}
% #############################################################################
% This is Appendix A
% !TEX root = ../main.tex
% #############################################################################
\chapter{Code of Project}
\label{chapter:appendixA}

Nulla dui purus, eleifend vel, consequat non, dictum porta, nulla. Duis ante mi, laoreet ut, commodo eleifend, cursus nec, lorem. Aenean eu est. Etiam imperdiet turpis. Praesent nec augue. Curabitur ligula quam, rutrum id, tempor sed, consequat ac, dui. Vestibulum accumsan eros nec magna. Vestibulum vitae dui. Vestibulum nec ligula et lorem consequat ullamcorper. 

\begin{lstlisting}[frame=lines,style=XML,caption={Example of a XML file.},label=xmlEx]
<?xml version="1.0" encoding="UTF-8"?>
<StreamInfo version="2.0">
    <Clip duration="PT01M0.00S">
        <BaseURL>videos/</BaseURL>
        <Description>svc_1</Description>
        <Representation mimeType="video/SVC" codecs="svc" frameRate="30.00" bandwidth="401.90"
            width="176" height="144" id="L0">
            <BaseURL>svc_1/</BaseURL>
            <SegmentInfo from="0" to="11" duration="PT5.00S">
                <BaseURL>svc_1-L0-</BaseURL>
            </SegmentInfo>
        </Representation>
        <Representation mimeType="video/SVC" codecs="svc" frameRate="30.00" bandwidth="1322.60"
            width="352" height="288" id="L1">
            <BaseURL>svc_1/</BaseURL>
            <SegmentInfo from="0" to="11" duration="PT5.00S">
                <BaseURL>svc_1-L1-</BaseURL>
            </SegmentInfo>
        </Representation>
    </Clip>
</StreamInfo>
\end{lstlisting}

Etiam imperdiet turpis. Praesent nec augue. Curabitur ligula quam, rutrum id, tempor sed, consequat ac, dui. Maecenas tincidunt velit quis orci. Sed in dui. Nullam ut mauris eu mi mollis luctus. Class aptent taciti sociosqu ad litora torquent per conubia nostra, per inceptos hymenaeos. Sed cursus cursus velit. Sed a massa. Duis dignissim euismod quam.

\begin{spacing}{0.5}
\lstinputlisting[style=coloredASM,language=Assembler,numbers=left,caption={Assembler Main Code.},label=code]
{./tables_and_code/example.asm.txt}
\end{spacing}


Class aptent taciti sociosqu ad litora torquent per conubia nostra, per inceptos hymenaeos. Phasellus eget nisl ut elit porta ullamcorper. Maecenas tincidunt velit quis orci. Sed in dui. Nullam ut mauris eu mi mollis luctus. Class aptent taciti sociosqu ad litora torquent per conubia nostra, per inceptos hymenaeos.

This inline MATLAB code \mcode{for i=1:3, disp('cool'); end;} uses the \verb|\mcode{}| command.\footnote{MATLAB Works also in footnotes: \mcodefn{for i=1:3, disp('cool'); end;}}

Nullam ut mauris eu mi mollis luctus. Class aptent taciti sociosqu ad litora torquent per conubia nostra, per inceptos hymenaeos. Sed cursus cursus velit. Sed a massa. Duis dignissim euismod quam. Nullam euismod metus ut orci.

\begin{lstlisting}[language=matlabfloz,caption={\mcode{Matlab Function}}]
for i = 1:3
	if i >= 5 && a ~= b       % literate programming replacement
		disp('cool');         % comment with some §\mcommentfont\LaTeX in it: $\mcommentfont\pi x^2$§
	end
	[:,ind] = max(vec);
	x_last = x(1,end) - 1;
	v(end);
	ylabel('Voltage (µV)');
end
\end{lstlisting}

Nullam ut mauris eu mi mollis luctus. Class aptent taciti sociosqu ad litora torquent per conubia nostra, per inceptos hymenaeos. Sed cursus cursus velit. Sed a massa. Duis dignissim euismod quam. Nullam euismod metus ut orci.

\lstinputlisting[
	label=lst:matlab_code,
	caption={\mcode{function.m}},
	breaklines=true
	]{./tables_and_code/function.m}

Class aptent taciti sociosqu ad litora torquent per conubia nostra, per inceptos hymenaeos. Phasellus eget nisl ut elit porta ullamcorper. Maecenas tincidunt velit quis orci. Sed in dui. Nullam ut mauris eu mi mollis luctus. Class aptent taciti sociosqu ad litora torquent per conubia nostra, per inceptos hymenaeos. Sed cursus cursus velit. Sed a massa. Duis dignissim euismod quam. Nullam euismod metus ut orci. Vestibulum erat libero, scelerisque et, porttitor et, varius a, leo.

\begin{lstlisting}[style=htmlcssjs,caption={HTML with CSS Code}]
<!DOCTYPE html>
<html>
  <head>
    <title>Listings Style Test</title>
    <meta charset="UTF-8">
    <style>
      /* CSS Test */
      * {
        padding: 0;
        border: 0;
        margin: 0;
      }
    </style>
    <link rel="stylesheet" href="css/style.css" />
  </head>
  <header> hey </header>
  <article> this is a article </article>
  <body>
    <!-- Paragraphs are fine -->
    <div id="box">			
			<p>
			  Hello World
			</p>
      <p>Hello World</p>
      <p id="test">Hello World</p>
			<p></p>
    </div>
    <div>Test</div>
    <!-- HTML script is not consistent -->
    <script src="js/benchmark.js"></script>
    <script>
      function createSquare(x, y) {
        // This is a comment.
        var square = document.createElement('div');
        square.style.width = square.style.height = '50px';
        square.style.backgroundColor = 'blue';
        
        /*
         * This is another comment.
         */
        square.style.position = 'absolute';
        square.style.left = x + 'px'; 
        square.style.top = y + 'px';
        
        var body = document.getElementsByTagName('body')[0];
        body.appendChild(square);
      };
      
      // Please take a look at +=
      window.addEventListener('mousedown', function(event) {
        // German umlaut test: Berührungspunkt ermitteln
        var x = event.touches[0].pageX;
        var y = event.touches[0].pageY;
        var lookAtThis += 1;
      });
    </script>
  </body>
</html>
\end{lstlisting}

Nulla dui purus, eleifend vel, consequat non, dictum porta, nulla. Duis ante mi, laoreet ut, commodo eleifend, cursus nec, lorem. Aenean eu est. Etiam imperdiet turpis. Praesent nec augue. Curabitur ligula quam, rutrum id, tempor sed, consequat ac, dui. Vestibulum accumsan eros nec magna. Vestibulum vitae dui. Vestibulum nec ligula et lorem consequat ullamcorper.

\begin{lstlisting}[style=htmlcssjs,caption={HTML CSS Javascript Code}]

@media only screen and (min-width: 768px) and (max-width: 991px) {
	
	#main {
		width: 712px;
		padding: 100px 28px 120px;
	}
	
	/* .mono {
		font-size: 90%;
	} */
	
	.cssbtn a {
		margin-top: 10px;
		margin-bottom: 10px;
		width: 60px;  
		height: 60px;   
		font-size: 28px;
		line-height: 62px;
	}
\end{lstlisting}

Nulla dui purus, eleifend vel, consequat non, dictum porta, nulla. Duis ante mi, laoreet ut, commodo eleifend, cursus nec, lorem. Aenean eu est. Etiam imperdiet turpis. Praesent nec augue. Curabitur ligula quam, rutrum id, tempor sed, consequat ac, dui. Vestibulum accumsan eros nec magna. Vestibulum vitae dui. Vestibulum nec ligula et lorem consequat ullamcorper.

\begin{lstlisting} [style=py,caption={PYTHON Code}]
class TelgramRequestHandler(object):
    def handle(self):
        addr = self.client_address[0]         # Client IP-adress
        telgram = self.request.recv(1024)     # Recieve telgram
        print "From: %s, Received: %s" % (addr, telgram)
        return
\end{lstlisting}
%% If Printing on DOUBLE SIDED pages, the second page should be white.
%% Otherwise, comment the following command:
\cleardoublepage
%% Second Appendix
\pdfbookmark[1]{Appendix B}{appendix}
\input{./Chapters/Thesis-MSc-AppendixB.tex}
%% If Printing on DOUBLE SIDED pages, the second page should be white.
%% Otherwise, comment the following command:
\cleardoublepage

% -----------------------------------------------------------------------------
% And this is THE END of the IST Thesis Document
\end{document}