% #############################################################################
% RESUMO em Português
% !TEX root = ../main.tex
% #############################################################################
% use \noindent in firts paragraph
\noindent 


\par A nível mundial, existe um aumento no interesse da execução de missões de complexidade crescentemente elevada que envolvem a utilização de vários veículos autónomos cooperativos sem supervisão constante de operadores humanos. Um fator chave para a execução de tais missões é a disponibilidade de métodos avançados para o planeamento de movimento cooperativo que leva explicitamente em conta restrições temporais e espaciais, limitações intrínsecas do veículo e requisitos de minimização de energia.
\par Motivadas pelas tendências atuais nesta área de pesquisa, as técnicas de planeamento de movimento aqui estudadas têm a tarefa de encontrar trajetórias viáveis e seguras para um grupo de veículos de modo a que atinjam vários pontos-alvo ao mesmo tempo (chamados "problemas de chegada simultânea") evitando colisões entre veículos, bem como entre veículos e obstáculos, tendo em conta restrições na energia consumida. Aqui, o problema de planeamento do movimento é formulado como um problema de controlo ótimo em tempo contínuo. Diferentes métodos numéricos diretos que aproximam variáveis de forma discreta são explorados com base em polinómios de Bernstein. Estes polinómios possuem propriedades convenientes que permitem o cálculo eficiente e aplicação de restrições ao longo das trajetórias dos veículos, como velocidade máxima, taxas angulares, distância mínima entre as trajetórias, bem como a distância mínima entre os veículos e obstáculos.
\par Serão avaliadas diferentes ferramentas matemáticas para calcular custos e viabilidade. Por último, são apresentados resultados de simulações que mostram a eficácia do algoritmo completo para números específicos de veículos e diferentes restrições.