% #############################################################################
% This is Chapter 6
% !TEX root = ../main.tex
% #############################################################################
% Change the Name of the Chapter i the following line
\fancychapter{Conclusion}
\cleardoublepage%
\label{chap:conclusion}

% #############################################################################

\par In this work, a fast optimal motion planning algorithm was designed. It consisted in solving an optimal control problem, by finding its equivalent optimisation problem. The algorithm's final form was chosen based on the comparison of several parameterisation methods and how well each one can handle dynamic and environmental constraints. Bezier curves was the final choice of parameterisation to approximate the optimal trajectory. They have very good properties for motion planning that greatly simplify the optimisation problem.

Two \ac{AUV} models, the unicycle and the Medusa, were studied and afterwards, how can their characteristics reap the benefits from the proposed motion planning algorithm. Specifically, it is shown that the use of a Bernstein based polynomial representation of trajectories is no longer limited for differentially flat systems and how, in fact, defining all of the state variables and inputs and linking them via the dynamics simplifies the computation of running costs and calculating feasibility of the solution.
\par Results also show, however, that time complexity quickly increases with a high number of vehicles, specially when a high order of approximation is used for every vehicle's state and input variables. These time complexity limitations could potentially be negligible if a more advanced computational power is available.
\par This algorithm, not only solves the problem for the go-to-formation maneuver, to start cooperative missions, but also allows to add other objectives such active navigation localization. The tools developed to plan optimal trajectories can now be explored to conduct further research. 

