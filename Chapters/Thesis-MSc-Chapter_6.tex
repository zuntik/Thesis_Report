% #############################################################################
% This is Chapter 6
% !TEX root = ../main.tex
% #############################################################################
% Change the Name of the Chapter i the following line
\fancychapter{Conclusion}
\cleardoublepage%
\label{chap:conclusion}

% #############################################################################
\par The project was carried out within the master's program Robotics and Control under the supervision of Professor Pascoal. It consisted of the development a fast optimal motion planning algorithm for finding feasible and safe trajectories for a group of vehicles such that they reach a number of target points at the same time. It involed solving an optimal control problem by finding its equivalent finite dimensional optimisation problem. The final form of the algorithm was chosen based on the comparison of several parameterisation methods and their performance in handling dynamics and environmental constraints. Bezier curves were the final choice of parameterisation to approximate the optimal trajectory due to their convenient properties that allow efficient computation and enforcement of constraints along the vehicles’ trajectories.



\par The motion planning algorithm was tested with two \ac{AMV} models: the Dubin's car model and the Medusa model, whose ruling kinematic and dynamics equations were also studied and presented here. It can be concluded that motion planning can now be performed for non-differentially flat systems such as the Medusa model.
\par Results of the application tests show how, with increasing order, the final cost quickly converges to optimal but at the expense of computation time. As a result, a trade-off must be found between optimality and the computation time when deploying the proposed algorithms in real life scenarios. It also can also be concluded that introducing an iterative algorithm in each step of the optimisation process - such as the presented \textit{minimum distance to a polygon algorithm} presented here - introduces a disproportionate amount of computation time.
\par A trade off must also be found between order and number of vehicles, because as it has been seen in the tests, the computation time can quickly grow when adding more vehicles, even when the fastest sample-based minimum distance algorithms are used.
\par The motion planning algorithm, not only solves the problem for the go-to-formation maneuver but also supports other kinds of missions such as those based on active navigation. Therefore, future research can be based on applying the algorithms presented here for a wide range of complex motion planning problems. Another further investigation that can be derived from this work is testing the motion planning algorithms in close cooperation with trajectory tracking such that there can be real applied in real time.



%\par In this work, a fast optimal motion planning algorithm was designed. It consisted in solving an optimal control problem, by finding its equivalent finite dimensional optimisation problem. The algorithm's final form was chosen based on the comparison of several parameterisation methods and how well each one can handle dynamic and environmental constraints. Bezier curves was the final choice of parameterisation to approximate the optimal trajectory. They have very good properties for motion planning that greatly simplify the optimisation problem.