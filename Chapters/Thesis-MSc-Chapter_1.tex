\fancychapter{Introduction}
\cleardoublepage%
% The following line allows to ref this chapter
\label{chap:intro}%



\section{Motivation}
 

\par Worldwide, there has been growing interest in the use of autonomous vehicles to execute missions of increasing complexity without constant supervision of human operators. 
\par The WiMUST project \cite{sonar_tec_overview} is an example of a project that demanded the use of such autonomous vehicles. The goal of the WiMUST project was to design a system of cooperating \acp{AMV} able to perform innovative geotechnical surveying operations. Specifically, the WiMUST system consisted of an array of physically disconnected \acp{AMV}, acting as intelligent sensing and communicating nodes of a moving acoustic network. Together, the vehicles form a geometry formation which is actively controllable according to the needs of a specific application. 
\par Applications for the \ac{AMV} formation handled by the WiMUST system include seabed mapping, seafloor characterization and seismic exploration. The overall system behaves as a distributed acoustic array capable of acquiring acoustic data obtained by illuminating the seabed and the ocean sub-bottom with strong acoustic waves sent by one (or more) acoustic sources installed onboard a support ship/boat (see figure \ref{fig:WiMUST_System}). Advantages of multiple \acp{AMV} acting cooperatively as opposed to a single one include robustness against failure of a single node and improving the seabed and sub-bottom resolution. They can also adapt better to unforeseen circumstances in the terrain by making better use of the larger environments that they can observe as the spatial distance between each \ac{AMV} can be varied. 
\par Another, unrelated, example of application that justifies the use of multiple autonomous vehicles was the Intel show at the 2018 Winter Olympics, where 1200 drones were used to put on a light show in the night sky. Each drone was mounted with a light bulb and, together, formed different shapes in the sky.
\par Complex systems like the two previously presented have in common the ability to properly plan motion for each vehicle that satisfies certain criteria such as simultaneous arrival of each vehicle in a formation while minimizing spent energy or time and avoiding collisions between vehicles and the environment.
\par Over the past decades, many approaches to solve motion planning problems have been proposed. Examples include bug algorithms, randomised algorithms such as PRM, RRT, RRT*, cell decomposition methods, graph-based approaches, planners based on learning, and methods based on optimal control formulations. Each technique has different advantages and disadvantages, and is best-suited for certain types of problems. Trajectory generation based on optimal control formulations stands out as particularly suitable for applications that require the trajectories to minimize (or maximize) some cost function while satisfying a complex set of vehicle and problem constraints. Finding closed-form solutions for Optimal Control problems can be difficult or even impossible to solve, and therefore numerical methods must be sought. Numerical methods can be divided into Direct and Indirect Methods. A numerical method for solving such complex Optimal Control problems consists in optimising trajectories given by Bernstein Polynomials. Recent pioneering work on direct methods based on Bernstein Polynomials \cite{lorentz2013bernstein} show interesting results \cite{cichella2018bernstein}, specifically, they can n can solve complex problems with a high number of vehicles while keeping computational time relatively low. The use of these methods will be further explored in the work of this thesis.

\begin{figure}[h!]
    \centering
    \includegraphics[width=0.5\textwidth]{Images/projects/WiMUST_project.jpg}
    \caption{Artist’s rendition of the WiMUST system for sub-bottom acoustic profiling with source-receiver decoupling}
    \label{fig:WiMUST_System}
\end{figure}


\section{Background}

\par The work discussed in this thesis focuses on motion planning for multiple cooperative vehicles. Motion Planning, as the name suggests, consists in planning motion for robots, such as mobile vehicles or robotic arms. Trajectory generation based on optimal control formulations will be used, specifically, to solve the motion planning problems. 
\par A trajectory is a time parameterized set of states of a dynamical system. These states can be position, their derivatives, heading, among others. The states and inputs are related to each other by a set of dynamic equations. When dealing with multiple vehicles, the trajectory optimisation problem takes into account the union of states and inputs of each vehicles such that the solution describes trajectories for all of the vehicles simultaneously.
\par As discussed in the previous section, numerical solutions to the trajectory generation problems must be sought. The numerical methods can be grouped into either direct methods or indirect methods. 
\par Indirect methods “use the necessary conditions of optimality of an infinite problem to derive a boundary value problem in ordinary differential equations”, the solutions of which must be found using analytic or numerical methods. 
\par Direct methods, on the other hand, are based on transcribing infinite optimal control problems into finite-dimensional \acp{NLP} using some kind of paramerisation (e.g., polynomial approximation or piecewise constant parameterization). They can be solved using ready-to-use NLP solvers (e.g. MATLAB) and do not require the computation of co-state and adjoint variables as indirect methods do.
\par The focus on the work of this thesis will be on the usage of Direct methods. Several parameterisation methods will be explored, such as peace-wise constants inputs, polynomials, and the usage of Bernstein polynomials.
\par When solving an optimisation problem, a model must be chosen for each vehicle. 
Different models will be defined by different numbers of states and inputs and different dynamic equations. The choice of model will affect the complexity of the optimisation problem, however, if the model is too simple, it may not accurately describe the real vehicle's dynamics. The Direct methods that will be tested will focus on two models, the Medusa Model \cite{abreu2016medusa} or the simpler Dubin's car \cite{Reeds1990OPTIMALPF}.


\section{Objectives}

\par The work presented in this thesis focuses on the usage of Direct Methods.
\par There are several direct methods for trajectory optimisation, for example, single and multiple shooting, collocation and quadratic programming. However, polynomial methods based on Bezier curves are particularly advantageous because they have favourable geometric properties which allow the efficient computation of the minimum distance between trajectories. As the complexity of the polynomials increases, the solutions converge to the optimal. \cite{cichella2018bernstein}
\par An Optimal Control Problem's cost can be constructed based on several criteria such as consumed energy. For \textit{cooperative} motion planning, the cost will have to be constructed differently because it will have to take into account the motion of the multiple vehicles at once, in particular, possible inter-vehicle collision.
\par In practice, some of the objectives of this work include
\begin{itemize}
    \item test some methods for obstacle avoidance
    \item compare different parameterisation methodologies
    \item analyse the complexity of increasing order and number of vehicles
    \item test viability for non differentially flat systems
    \item test the usage of log barrier functions that may help with speeding up the optimisation process because they reduce the number of constraints by placing them in the cost function.
\end{itemize}


\section{Thesis Outline}

This thesis is organized as follows.  In Chapter \ref{chap:theory}, a number of direct optimisation methods are presented, together with a simple motivating example that compares these methods. Chapter \ref{chap:autonomousvehiclemodels} introduces the two autonomous marine vehicle models that were used along with the simplifications adopted appropriate for the motion planning algorithms that were developed. Chapter \ref{chap:implementation} presents a formulation of the motion planning problem in the form of an equivalent optimisation problem and the mathematical tools to solve it. Some results for particular motion planning problems are presented in chapter \ref{chap:results}. Finally, chapter \ref{chap:conclusion} summarizes the main achievements of the thesis and discusses topics that warrant future work.

