\noindent

\par Worldwide, there has been growing interest in the execution of missions of increasing complexity involving the use of several autonomous vehicles acting cooperatively without constant supervision of human operators. A key enabling factor for the execution of such missions is the availability of advanced methods for cooperative motion planning that take explicitly into account temporal and spatial constraints, intrinsic vehicle limitations and energy minimization requirements.
\par Motivated by the current trends in this area of research, the motion planning techniques studied here are tasked with finding feasible and safe trajectories for a group of vehicles such that they reach a number of target points at the same time (the so-called "simultaneous arrival problem") and avoid inter-vehicle as well as vehicle/obstacle collisions, subject to the constraint that the overall energy required for vehicle motion is minimized.
Here, the motion planning problem is formulated as a continuous-time optimal control problem. Different numerical Direct Methods that approximate its solutions in a discretized setting are explored with a focus on Bernstein polynomials. These polynomials possess convenient properties that allow for efficient computation and enforcement of constraints along the vehicles’ trajectories, such as maximum speed, angular rates, and the minimum distance between trajectories along with the minimum distance between the vehicles and known obstacles.
\par Different mathematical tools to calculate cost and feasibility are evaluated. Finally, the results of simulations aimed at showing the efficacy of the complete motion planning algorithm developed for specific numbers of vehicles and different constraints are presented.